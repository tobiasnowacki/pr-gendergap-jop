\begin{table}

\caption{\label{tab:poland_main} \textbf{Difference-in-Discontinuity Estimates For Incumbency Advantage In Polish Counties and County-Like Cities.} The gender gap is similar in magnitude to that of Norwegian municipalities.}
\centering
\fontsize{9}{11}\selectfont
\begin{threeparttable}
\begin{tabular}[t]{lS[
              input-symbols=(),
              table-format=-1.3,
              table-space-text-pre    = (,
              table-space-text-post   = ),
              input-open-uncertainty  =,
              input-close-uncertainty = ,
              table-align-text-post = false]S[
              input-symbols=(),
              table-format=-1.3,
              table-space-text-pre    = (,
              table-space-text-post   = ),
              input-open-uncertainty  =,
              input-close-uncertainty = ,
              table-align-text-post = false]S[
              input-symbols=(),
              table-format=-1.3,
              table-space-text-pre    = (,
              table-space-text-post   = ),
              input-open-uncertainty  =,
              input-close-uncertainty = ,
              table-align-text-post = false]S[
              input-symbols=(),
              table-format=-1.3,
              table-space-text-pre    = (,
              table-space-text-post   = ),
              input-open-uncertainty  =,
              input-close-uncertainty = ,
              table-align-text-post = false]S[
              input-symbols=(),
              table-format=-1.3,
              table-space-text-pre    = (,
              table-space-text-post   = ),
              input-open-uncertainty  =,
              input-close-uncertainty = ,
              table-align-text-post = false]S[
              input-symbols=(),
              table-format=-1.3,
              table-space-text-pre    = (,
              table-space-text-post   = ),
              input-open-uncertainty  =,
              input-close-uncertainty = ,
              table-align-text-post = false]}
\toprule
\multicolumn{1}{c}{ } & \multicolumn{3}{c}{Run (t + 1)} & \multicolumn{3}{c}{Win (t + 1)} \\
\cmidrule(l{3pt}r{3pt}){2-4} \cmidrule(l{3pt}r{3pt}){5-7}
  & \multicolumn{1}{c}{(1)} & \multicolumn{1}{c}{(2)} & \multicolumn{1}{c}{(3)} & \multicolumn{1}{c}{(4)} & \multicolumn{1}{c}{(5)} & \multicolumn{1}{c}{(6)}\\
\midrule
Elected & 0.109 & 0.110 & 0.122 & 0.134 & 0.121 & 0.157\\
 & (0.032) & (0.026) & (0.043) & (0.030) & (0.023) & (0.042)\\
\addlinespace
Female & -0.026 & -0.053 & -0.045 & -0.001 & -0.027 & 0.054\\
 & (0.047) & (0.037) & (0.056) & (0.035) & (0.027) & (0.049)\\
\addlinespace
Elected x Female & 0.030 & 0.048 & 0.001 & -0.069 & -0.057 & -0.116\\
 & (0.065) & (0.053) & (0.081) & (0.057) & (0.044) & (0.078)\\
\addlinespace \midrule \addlinespace
Bandwidth & 0.065 & 0.13 & 0.033 & 0.059 & 0.12 & 0.029\\
BW Type & \multicolumn{1}{c}{Optimal} & \multicolumn{1}{c}{2x Opt} & \multicolumn{1}{c}{0.5x Opt} & \multicolumn{1}{c}{Optimal} & \multicolumn{1}{c}{2x Opt} & \multicolumn{1}{c}{0.5x Opt}\\
Outcome Mean & 0.369 & 0.374 & 0.365 & 0.155 & 0.163 & 0.153\\
N (left) & \multicolumn{1}{c}{1227} & \multicolumn{1}{c}{1681} & \multicolumn{1}{c}{827} & \multicolumn{1}{c}{1160} & \multicolumn{1}{c}{1619} & \multicolumn{1}{c}{772}\\
N (right) & \multicolumn{1}{c}{1221} & \multicolumn{1}{c}{1673} & \multicolumn{1}{c}{822} & \multicolumn{1}{c}{1154} & \multicolumn{1}{c}{1611} & \multicolumn{1}{c}{767}\\
\bottomrule
\end{tabular}
\begin{tablenotes}[para]
\item All estimates are reported with robust standard errors clustered at the municipality level in parentheses. Each observation is a candidate's election attempt. 'Elected' is an indicator for observations where the candidate obtained a seat in the municipal council. 'Female' is an indicator for observations identified as female. `Elected` times `Female` is the interaction between the two variables. Other coefficients not reported. Regression run on all candidates from two main parties (PO, PiS) in county-level elections in 2010.
\end{tablenotes}
\end{threeparttable}
\end{table}